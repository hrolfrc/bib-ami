\documentclass[11pt, a4paper]{article}

% --- PREAMBLE: PACKAGES AND SETUP ---
\usepackage[utf8]{inputenc}
\usepackage[T1]{fontenc}
\usepackage{geometry}
\usepackage{booktabs} 
\usepackage{enumitem} 
\usepackage{hyperref} 
\usepackage{listings}
\usepackage{xcolor}

% --- GEOMETRY ---
\geometry{a4paper, margin=1in}

% --- HYPERREF SETUP ---
\hypersetup{
    colorlinks=true,
    linkcolor=blue,
    filecolor=magenta,      
    urlcolor=cyan,
    pdftitle={bib-ami Documentation},
    pdfauthor={bib-ami Team},
}

% --- LISTINGS (CODE) SETUP ---
\definecolor{codegray}{rgb}{0.5,0.5,0.5}
\definecolor{codepurple}{rgb}{0.58,0,0.82}
\definecolor{backcolour}{rgb}{0.95,0.95,0.92}

\lstdefinestyle{mystyle}{
    backgroundcolor=\color{backcolour},   
    commentstyle=\color{codegray},
    keywordstyle=\color{magenta},
    numberstyle=\tiny\color{codegray},
    stringstyle=\color{codepurple},
    basicstyle=\ttfamily\footnotesize,
    breakatwhitespace=false,         
    breaklines=true,                 
    captionpos=b,                    
    keepspaces=true,                 
    numbers=left,                    
    numbersep=5pt,                  
    showspaces=false,                
    showstringspaces=false,
    showtabs=false,                  
    tabsize=2
}
\lstset{style=mystyle}


% --- TITLE ---
\title{\textbf{bib-ami}: A Bibliography Integrity Manager \\ \large Part 4: Advanced Usage \& Configuration}
\author{Documentation}
\date{\today}


% ==================================================================
% --- DOCUMENT START ---
% ==================================================================
\begin{document}
\maketitle
\tableofcontents
\newpage

\section{Usage \& Configuration}

This section serves as a reference guide for users who want to customize the behavior of \texttt{bib-ami}. It provides a detailed description of all available command-line arguments and explains how to use a configuration file for persistent settings.

\subsection{Command-Line Reference}

The \texttt{bib-ami} tool is controlled via a set of command-line arguments. The basic syntax is as follows:

\begin{lstlisting}[language=bash, caption={General syntax for bib-ami}]
bib-ami --input-dir <path> --output-file <path> [options]
\end{lstlisting}

Below is a complete list of all available options.

\begin{description}[leftmargin=*,labelindent=1cm]
    \item[\texttt{--input-dir <path>}] \hfill \\
    \textbf{(Required)} Specifies the path to the directory containing the source \texttt{.bib} files that you want to process. The tool will scan this directory for all files ending with the \texttt{.bib} extension.

    \item[\texttt{--output-file <path>}] \hfill \\
    \textbf{(Required)} Specifies the full path, including the filename, for the main cleaned and verified bibliography file. For example: \texttt{output/final\_library.bib}.

    \item[\texttt{--suspect-file <path>}] \hfill \\
    (Optional) Specifies the path for a separate file where all 'Suspect' entries will be saved. A suspect entry is one that could not be verified (e.g., a modern journal article with no DOI). This option is most effective when used with \texttt{--filter-validated}.

    \item[\texttt{--merge-only}] \hfill \\
    (Optional) A flag that instructs the tool to perform only the first phase of the workflow. It will merge all source \texttt{.bib} files into the specified output file without performing any deduplication, validation, or enrichment.

    \item[\texttt{--email <address>}] \hfill \\
    (Optional) Specifies the email address to be used for querying the CrossRef API. Providing an email is part of CrossRef's "Polite Pool" policy, which helps them contact you if your script causes issues and may result in more reliable API access. This argument overrides any email set in the configuration file.

    \item[\texttt{--filter-validated}] \hfill \\
    (Optional) A flag that changes the output behavior. When this flag is active, only entries that are fully 'Verified' (i.e., have a trusted DOI and refreshed metadata) will be saved to the main \texttt{--output-file}. All other entries ('Accepted' and 'Suspect') will be saved to the file specified by \texttt{--suspect-file}. If \texttt{--suspect-file} is not provided, non-verified entries will be commented out in the main output file.
\end{description}

\subsection{Configuration File}

For settings that you use frequently, such as your email address, you can create a configuration file to avoid typing the same argument repeatedly.

\begin{enumerate}[leftmargin=*]
    \item Create a file named \texttt{bib\_ami\_config.json} in the directory from which you are running the \texttt{bib-ami} command.
    \item Add your settings to the file in JSON format. Currently, only the email address is supported.
\end{enumerate}

\begin{lstlisting}[language=json, caption={Example bib\_ami\_config.json file}]
{
  "email": "your.name@university.edu"
}
\end{lstlisting}

\subsubsection{Order of Precedence}
The tool uses the following order of precedence for settings:
\begin{enumerate}
    \item A command-line argument (e.g., \texttt{--email}) will \textbf{always override} any other setting.
    \item If a command-line argument is not provided, the tool will look for the setting in the \texttt{bib\_ami\_config.json} file.
    \item If the setting is found in neither location, a default value may be used, or the user may be prompted if the setting is required.
\end{enumerate}

\end{document}
