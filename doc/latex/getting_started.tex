\documentclass[11pt, a4paper]{article}

% --- PREAMBLE: PACKAGES AND SETUP ---
\usepackage[utf8]{inputenc}
\usepackage[T1]{fontenc}
\usepackage{geometry}
\usepackage{booktabs}
\usepackage{enumitem}
\usepackage{hyperref}
\usepackage{listings}
\usepackage{xcolor}

% --- GEOMETRY ---
\geometry{a4paper, margin=1in}

% --- HYPERREF SETUP ---
\hypersetup{
    colorlinks=true,
    linkcolor=blue,
    filecolor=magenta,
    urlcolor=cyan,
    pdftitle={bib-ami Documentation},
    pdfauthor={bib-ami Team},
}

% --- LISTINGS (CODE) SETUP ---
\definecolor{codegray}{rgb}{0.5,0.5,0.5}
\definecolor{codepurple}{rgb}{0.58,0,0.82}
\definecolor{backcolour}{rgb}{0.95,0.95,0.92}

\lstdefinestyle{mystyle}{
    backgroundcolor=\color{backcolour},
    commentstyle=\color{codegray},
    keywordstyle=\color{magenta},
    numberstyle=\tiny\color{codegray},
    stringstyle=\color{codepurple},
    basicstyle=\ttfamily\footnotesize,
    breakatwhitespace=false,
    breaklines=true,
    captionpos=b,
    keepspaces=true,
    numbers=left,
    numbersep=5pt,
    showspaces=false,
    showstringspaces=false,
    showtabs=false,
    tabsize=2
}
\lstset{style=mystyle}


% --- TITLE ---
\title{\textbf{bib-ami}: A Bibliography Integrity Manager}
\author{Documentation}
\date{\today}


% ==================================================================
% --- DOCUMENT START ---
% ==================================================================
\begin{document}
\maketitle
\tableofcontents
\newpage

\section{Introduction}

\textbf{bib-ami} is a command-line utility for improving the integrity of BibTeX (\texttt{.bib}) bibliographies. It is designed to help researchers, academics, and students by automating several common data cleaning tasks.

Managing bibliographies often involves dealing with duplicate entries, inconsistent formatting, and missing metadata. \textbf{bib-ami} addresses these issues by providing functionality to merge multiple \texttt{.bib} files, identify and remove duplicate entries, and validate or find missing Digital Object Identifiers (DOIs) by querying the CrossRef API. The goal is to produce a consolidated and more reliable BibTeX file, reducing the manual effort required to manage academic references.

\section{Getting Started}

This section is designed to get you up and running with \textbf{bib-ami} in under five minutes.

\subsection{Installation}
\textbf{bib-ami} is distributed via the Python Package Index (PyPI) and can be installed easily using \texttt{pip}. Ensure you have Python 3.7 or higher installed.

\begin{lstlisting}[language=bash, caption={Installing bib-ami via pip}]
pip install bib-ami
\end{lstlisting}

This command will also install all necessary dependencies, including \texttt{bibtexparser}, \texttt{requests}, and \texttt{fuzzywuzzy}.

\subsection{Quick Start Example}
The fastest way to see \textbf{bib-ami} in action is to run it on a directory of your existing \texttt{.bib} files. This single command will merge all found files, deduplicate the entries, attempt to find missing DOIs, and save the result to a new, clean file.

\begin{enumerate}[leftmargin=*]
    \item Create a directory (e.g., \texttt{my\_bib\_files}) and place all your source \texttt{.bib} files inside it.
    \item Create a second directory for the output (e.g., \texttt{output}).
    \item Run the following command from your terminal, making sure to provide your email address. An email is required for responsible use of the CrossRef API's Polite Pool.
\end{enumerate}

\begin{lstlisting}[language=bash, caption={A typical command to clean a bibliography}]
bib-ami --input-dir my_bib_files --output-file output/cleaned_library.bib --email "your.name@university.edu"
\end{lstlisting}

After the process completes, you will find a \texttt{cleaned\_library.bib} file in your \texttt{output} directory. This file contains the consolidated and enhanced collection of your references. A summary of the actions taken (e.g., duplicates removed, DOIs added) will be printed to your console.

\end{document}
