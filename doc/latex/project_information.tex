\documentclass[11pt, a4paper]{article}

% --- PREAMBLE: PACKAGES AND SETUP ---
\usepackage[utf8]{inputenc}
\usepackage[T1]{fontenc}
\usepackage{geometry}
\usepackage{booktabs} 
\usepackage{enumitem} 
\usepackage{hyperref} 
\usepackage{listings}
\usepackage{xcolor}

% --- GEOMETRY ---
\geometry{a4paper, margin=1in}

% --- HYPERREF SETUP ---
\hypersetup{
    colorlinks=true,
    linkcolor=blue,
    filecolor=magenta,      
    urlcolor=cyan,
    pdftitle={bib-ami Documentation},
    pdfauthor={bib-ami Team},
}

% --- LISTINGS (CODE) SETUP ---
\definecolor{codegray}{rgb}{0.5,0.5,0.5}
\definecolor{codepurple}{rgb}{0.58,0,0.82}
\definecolor{backcolour}{rgb}{0.95,0.95,0.92}

\lstdefinestyle{mystyle}{
    backgroundcolor=\color{backcolour},   
    commentstyle=\color{codegray},
    keywordstyle=\color{magenta},
    numberstyle=\tiny\color{codegray},
    stringstyle=\color{codepurple},
    basicstyle=\ttfamily\footnotesize,
    breakatwhitespace=false,         
    breaklines=true,                 
    captionpos=b,                    
    keepspaces=true,                 
    numbers=left,                    
    numbersep=5pt,                  
    showspaces=false,                
    showstringspaces=false,
    showtabs=false,                  
    tabsize=2
}
\lstset{style=mystyle}


% --- TITLE ---
\title{\textbf{bib-ami}: A Bibliography Integrity Manager \\ \large Part 6: Project Information}
\author{Documentation}
\date{\today}


% ==================================================================
% --- DOCUMENT START ---
% ==================================================================
\begin{document}
\maketitle
\tableofcontents
\newpage

\section{Project Information}

This final section provides supplementary information about the \texttt{bib-ami} project, including its software dependencies, guidance on how to cite the tool in your own work, and details on contributing.

\subsection{Dependencies}

The \texttt{bib-ami} tool relies on a small set of well-maintained, open-source Python packages. These will be installed automatically when you install \texttt{bib-ami} via \texttt{pip}.

\begin{description}[leftmargin=*]
    \item[\texttt{bibtexparser}] For robust parsing and writing of BibTeX files.
    \item[\texttt{requests}] For handling all HTTP requests to the CrossRef API.
    \item[\texttt{fuzzywuzzy}] For performing the fuzzy string matching required for deduplication.
    \item[\texttt{python-Levenshtein}] An optional but highly recommended C-based library that significantly speeds up the string similarity calculations used by \texttt{fuzzywuzzy}.
\end{description}

\subsection{How to Cite bib-ami}

If you use \texttt{bib-ami} in your research workflow and would like to cite it, please use the following BibTeX entry. Citing software helps the developers and maintainers justify their work and secure funding for future development.

\begin{lstlisting}[language=tex, caption={BibTeX entry for bib-ami}]
@misc{bib_ami_2024,
  author       = {Rolf Carlson},
  title        = {bib-ami: A BibTeX Integrity Manager},
  year         = {2025},
  publisher    = {GitHub},
  url          = {https://github.com/hrolfrc/bib-ami},
  doi          = {10.5281/zenodo.15795718}
}
\end{lstlisting}

\subsection{Contributing \& License}

\subsubsection{Contributing}
This project is open source and contributions are welcome. If you would like to contribute, please visit the project's GitHub repository. You can contribute by reporting bugs, suggesting new features, or submitting pull requests with code improvements. Please consult the \texttt{CONTRIBUTING.md} file in the repository for specific guidelines.

\subsubsection{License}
The \texttt{bib-ami} tool is distributed under the MIT License. This is a permissive open-source license that allows for broad use, modification, and distribution. Please see the \texttt{LICENSE} file in the source repository for the full text of the license.

\end{document}
